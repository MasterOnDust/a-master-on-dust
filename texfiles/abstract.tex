\chapter*{Abstract}
The aeolian dust deposits over the \acrfull{clp} contain valuable information about past environmental changes in Asia. Unlocking this information requires understanding the \acrfull{eadc} and, particularly, the components of the \acrshort{eadc} regarding the emissions, transport and deposition of aeolian dust over the \acrshort{clp}. 
In this work, the dust emission model FLEXDUST and the \acrfull{flexpart} were employed to study the dust cycle over the \acrshort{clp} (referred to as the \acrshort{eadc}) from 1999 until 2019, during the dust event season March until May.
Backward dust transport trajectories were calculated using \acrshort{flexpart} from seven sites across the \acrshort{clp}. Two particle size bins \SI{1.7}{\micro\metre}-\SI{2.5}{\micro\metre} representing clay particles and  \SI{15}{\micro\metre}-\SI{20}{\micro\metre} representing silt particles were included to examine the influence of particle size on dust transport and deposition, with wet and dry deposition being simulated, respectively.
The trajectory information from FLEXPART is combined with the dust emission field from FLEXDUST to produce a high-resolution map of the source contribution for each deposition site. 

The FLEXPART/FLEXDUST modelling setup was found to reasonably represent the \acrshort{eadc} compared to observations, especially its spatial and temporal variations. 
The main source of the dust deposited to \acrshort{clp} is the deserts to the northwest of the \acrshort{clp}. 
Wet deposition is the major contributor to the deposition of dust particles over the \acrshort{clp}, especially for the coarse silt particles through high-level tropospheric transport. 
An extensive correlation analysis between spring dust emissions, deposition and several climate indices reveals a strong correlation of dust deposition over the \acrshort{clp} with the winter \acrshort{ao}, but not with the \acrshort{eawm}. 
This contrasts with previous studies presuming a strong control of the \acrshort{eawm} on the dust deposition over the \acrshort{clp}.
Our results show that strong dust deposition years were characterised by a negative \acrshort{ao} and an anomaly in the winter 500hPa geopotential height over Mongolia/Siberia. 
The location of the geopotential height anomaly was consistent with previous studies on the \acrfull{caob} paths, indicating a key role of \acrshort{ao} in regulating the frequency, intensity and path of \acrshort{caob} intruding into East Asia, which primarily causes dust events in East Asia.



