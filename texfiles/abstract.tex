\chapter*{Abstract}
The aeolian dust deposits in the Chinese Loess Plateau (CLP) contain valuable information about past environmental changes in Asia. Unlocking this information requires knowledge on the Asian dust sources and dust transport mechanisms, and how the different source regions contribute to the total dust loading and deposition over the CLP.  By studying the dust transport and deposition under present day conditions using the Lagrangian Particle Dispersion model,  FLEXPART,  and the FLEXDUST dust emission model, we aim to better understand the dust signal in the Chinese loess records to constrain their interpretation as paleoclimate proxies.  

Here we present results from a 20 year simulation of transport and deposition of aeolian dust over the CLP from 1999 until 2019, during the dust event season March until May. Both FLEXPART and FLEXDUST are driven by ERA5 ECMWF meteorological reanalysis data. FLEXPART is set up in a receptor oriented configuration, where many computational particles are released from the receptor points at each timestep. The computational particles are followed for 5 days backward in time probing for possible source regions. The end product is emission sensitivity, i.e. how sensitive the receptor is to emissions in possible source regions. The emission sensitivity establishes a linear relation between the source and receptor. Therefore, multiplying the emission sensitivity with the dust emission flux estimated by FLEXDUST produces a map of the source contribution for each receptor point. To investigate the difference in source regions between the fine and coarse dust, we include two particle sizes, 2 μm and 20 μm, in our simulation.The output from the model is compared against Asian polar vortex (APV) and Asian winter monsoon indices to identify how changes in the large scale atmospheric circulation affect the interannual variation of dust transport and deposition, and to determine whether the amount of deposited dust over the CLP is primarily governed by changes in the emission strength or by changes in the atmospheric circulation.  
