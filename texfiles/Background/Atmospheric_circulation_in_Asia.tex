\section{Atmospheric circulation in East Asia}
The most prominent feature of the atmospheric circulation in the East Asian region is the monsoon system. The primary driver of the East Asian monsoon system is the temperature contrast between the land and ocean. During the winter months the land is generally colder than the ocean where the main heating source is located over the equatorial western pacific. The convection associated with the heating drives a large scale meridional overturning circulation where the airmasses rises over the ocean and then subsides over the Sebrian region. Creating a persistent high pressure system over Sebria during the winter. This circulation is the East Asian Winter Monsoon (EAWM). During the East Asian Summer Monsoon (EASM) the situation is reversed. In the summer the land is much warmer than the ocean, the heating on land drives strong convection which creates a persistent low pressure at the surface. The low pressure drags moist air from the ocean inland. Upon reaching the Tibetan Plateau the moist air is lifted and consequently cooled which causes the water vapor to condense resulting in intense precipitation. 

In following section will describe the EAWM and EASM circulation and how they are influenced by ...   

\subsection{East Asian Winter Monsoon}
The EAWM dominates the climate in East Asia during winter months, December, January, February, and is closely related to the cold core of the Siberian High.  

