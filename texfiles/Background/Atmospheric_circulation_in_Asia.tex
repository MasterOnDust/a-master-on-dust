\section{Atmospheric circulation in Asia}
The most prominent feature of the atmospheric circulation in Asian is the monsoon system. The monsoon system exist a as consequence of the uplifting of the high Tibetan Plateau, which influenced the atmospheric circulation by posing as a physical obstacle to the atmospheric flow and changing the pressure fields by altering the surface heating \parencite{molnar_orographic_2010}. The moonson system produces distinct seasonal cycle with wet spring/summer and dry winter. The Asian monsoon is typically divided into two broader monsoon systems, the South Asian monsoon and the East Asian monsoon (EAM). The South Asian monsoon includes the monsoon regions of India, the Indochina Peninsula and the South China sea, is a classical tropical monsoon circulation where most of the rain fall occur in the intertropical convergence zone (ITCZ), but will not be described further as it not key for understanding the influence of atmospheric circulation on dust emission and deposition. The EAM which is influences the climates of China, Korea and Japan is fundamentally different from the South Asian monsoon and is purely extratropical in nature, where the winds and precipitation are governed by frontal systems and the jet stream, as such one might argue that "monsoon" is a bit of a misnomer.

\subsection{East Asian Winter Monsoon}
