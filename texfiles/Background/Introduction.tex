\Chapter{}{Introduction}
 
 Mineral dust is an important component in the climate system. In atmosphere the dust directly affect earth's radiative balance by scattering and absorbing incoming radiation from the sun. Dust also plays an important role in cloud formation through acting as cloud condensation nuclei (CCN) or ice nuclei (IN) and through which it has an indirect impact on climate. However impact of dust on climate is not only limited to when it reside in the atmosphere. The dust will eventually settle, for instance if the dust where to be deposited in the ocean it can provide crucial nutrients for the ecosystems, or if the dust is deposited on a glacier, the dust could act to accelerate the melting through altering the surface albedo. Moreover the deserts them self are prone to climate change, e.g. changes in precipitation,local winds patterns or temperature are all factors that would affect the amount of emitted dust. This leads to complex dust climate feedbacks. The complexity and interconnections between dust and climate makes it difficult to confidently predict how dust would respond to climate change. 
 
 During the time frame millions of years dust has emitted from the desert regions of East Asia and deposited over the Chinese Loess Plateau (CLP). Creating the Chinese loess deposits which are the oldest and thickest deposits of Cenozoic aeolian dust. Advancing the understanding on how the loess deposits evolved would improve our knowledge about the past variability of East Asian climate, as well as assisting in creating more accurate predictions on future changes of east asian dust. 
 
In this thesis the inter-annual variation in dust sources, deposition and transport during 1999-2019 were investigated using the lagrangian particle dispersion model FLEXPART \parencite{Flexpart10.4_ref} and dust emission model FLEXDUST. The goal of this work is to improve our understanding of the controlling factors of the dust deposition over CLP, to ultimately aid the interpretation of the CLP records as an archive on environmental change.
 
Changes in loess grain size and mass accumulation rate (MAR) are typically attributed to changes in the winter monsoon strength. A strong winter monsoon is usually associated with a increased MAR. \textcite{wyrwoll2016cold} used measurements of visibility from several meteorological stations in china and the frequency of cold wave outbreaks. They found a strong correlation between cold wave out breaks and dust storm frequency. However the dust strom frequency, based on visibility measurements are more analogous to concentration. Concentration and deposition are not necessarily linked, in particular during wet deposition events. Second \textcite{wyrwoll2016cold} only considered the frequency of the dust events, but it is possible that single strong dust events contribute majority of the dust deposited. 

Given the weakness in prevous studies 
\par Mineral dust entrained into the atmosphere from the arid regions of East Asia has a major impact on both regional and global climate. However the mechanisms 