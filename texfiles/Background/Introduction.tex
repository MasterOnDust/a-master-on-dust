\Chapter{}{Introduction}
% Need to introduce the word aeolian. 

Aeolian transport of mineral dust has been a topic of research for decades. Emission of soil dust is a natural processes that has been occurring in the  
The stratigraphy of the \acrfull{clp} contains value information about past climate in East Asia, that has been encoded through the deposition of aeolian dust over a the time frame of millions of years. Forming the oldest and thickest deposits of Cenozoic aeolian dust.  

%Much information on dust sources and deposition amounts can also be retrieved from snow samples and ice cores. Especially in combination with transport modeling, these help to recognize important dust source regions. In ice core studies, a seasonal variation of dust deposition has been detected early [e.g., Hamilton and Langway, 1967], and this characteristic has been used to date ice core layers [e.g., Ram and Koenig, 1997].

Mineral dust plays an important role the earth system via influencing the radiative balance of the atmosphere \parencite{choobari2014global,sun2012numerical, yin_interactions_2002}, biogeochemicalcycles \parencite{jickells_dust_biogeo_2015} and local air quality.   
 
 Advancing the understanding on how the loess deposits evolved would improve our knowledge about the past variability of East Asian climate, as well as assisting in creating more accurate predictions on future changes of east asian dust. 

%Combinations of aerosols sample studies and trajectory analysis can provide additional confidence in determining source areas. Such a 

The aim of this thesis is to investigate inter-annual variations in transport and deposition of aeolian dust over the \acrshort{clp} and distinguish contributions
of the surrounding source regions.

In this thesis the inter-annual variation in dust sources, deposition and transport during 1999-2019 were investigated using the \acrfull{lpdm} FLEXPART \parencite{Flexpart10.4_ref} and dust emission model FLEXDUST. The goal of this work is to improve our understanding of the controlling factors of the dust deposition over CLP, to ultimately aid the interpretation of the CLP records as an archive on environmental change.
 
Changes in loess grain size and \acrfull{mar} are typically attributed to changes in the winter monsoon strength. A strong winter monsoon is usually associated with a increased \acrshort{mar}. \textcite{wyrwoll2016cold} used measurements of visibility from several meteorological stations in china and the frequency of cold wave outbreaks. They found a strong correlation between cold wave outbreaks and dust storm frequency. However the dust storm frequency, based on visibility measurements are more analogous to concentration. Concentration and deposition are not necessarily linked, in particular during wet deposition events. Second \textcite{wyrwoll2016cold} only considered the frequency of the dust events, but it is possible that single strong dust events contribute majority of the dust deposited. 

Given the weakness in prevous studies 
\par Mineral dust entrained into the atmosphere from the arid regions of East Asia has a major impact on both regional and global climate. However the mechanisms 