\thispagestyle{plain}
\Chapter{}{Introduction}
% Need to introduce the word aeolian. 
The windblown sediments at the \acrfull{clp} compromise the largest and thickest deposit of Cenozoic windblown dust. The \acrshort{clp} covers an area of approximately \SI{275600}{\kilo\metre\squared} and is at the thickest more than 200\si{\metre} thick \todo{reference}.
Dust from the proximal source regions has been naturally deposited over this region for millennials, and faithfully recording the past environmental history of the East Asia region. With the oldest parts of the Chinese loess sequences extending back 22Ma \parencite{qiang2011new}. The stratigraphy of the Quaternary \acrshort{clp} deposits is composed of layers of alternating loess and palaeosol sequences. The palaeosol sequences are linked to the interglacial periods during which East Asia experienced a more humid climate with increased soil formation connected to a stronger East Asian summer monsoon \todo{reference}. The loess sequences are thought to be formed during the glacial periods, which are thought to have been dominated by a stronger winter monsoon resulting in increased aridity and dust deposition \todo{reference}. Thus the Chinese loess serves as an important record of Quaternary glacial and interglacial periods in East Asian \todo{reference}.

In reconstruction studies of the \acrfull{eawm} changes in 
\acrfull{mar} and loess grain-size are typically attributed to variations in the \acrshort{eawm} strength \parencite{stevens2007reinterpreting}. Emerging from these studies has the general claim that there is a strengthening of the EAWM during glacial stages causing an increase in \acrshort{mar} with a shift towards larger grain sizes and conversely a significant downturn of \acrshort{eawm} activity during interglacial stages.
However, this interpretation has been cast under doubt because modern dust emissions in East Asia primarily a spring phenomena \parencite{sun2001spatial}.
Moreover, the possible mechanisms linking \acrshort{eawm} to springtime emissions remain an open question \parencite{roe2009interpretation}. 
\textcite{wyrwoll2016cold} investigated the linkage between modern dust emissions and the \acrshort{eawm} by examining the interannual correlations between visibility observations from several meteorological stations across China and the frequency of cold wave outbreaks over a 40 year period. In their study, the cold wave frequency was used as a proxy for the \acrshort{eawm} strength. 
The cold wave frequency was found to be strongly correlated to the dust storm frequency.
Conversely, \textcite{wyrwoll2016cold} concluded that the \acrshort{clp} serves as a valid record for the variability in the \acrshort{eawm} strength. However, the loess is formed through dust deposition, the visibility on the contrary is primarily an indicator of the atmospheric dust concentration. Dust concentration and deposition are not necessarily coupled, for instance during wet deposition episodes \todo{reference}. Moreover, \textcite{wyrwoll2016cold} only considered the frequency of the dust events, however single strong dust events may contribute to the majority of the dust deposited. Thus the motivation behind this thesis is to remedy the weaknesses of \textcite{wyrwoll2016cold} and examine whether the \acrshort{mar} is really linked to \acrshort{eawm} and if the grain-size would be a better indicator of \acrshort{eawm} strength. 
In this thesis, \acrfull{flexpart} \parencite{Flexpart10.4_ref} and dust emission model FLEXDUST \parencite{flexdust_ref_2016} are employed for the first time to simulate desert dust mobilisation, transport and deposition in East Asia \acrshort{clp}. In contrast to conventional integrated dust models, this model setup can be run backwards in time following the dust from the receptor site to the source regions, making it possible to establish a detailed map of how much each source element contributes to the deposition. The inter-annual variations in springtime dust emissions, transport and deposition over \acrshort{clp} are examined over a 20 year period during 1999-2019. The backward FLEXPART simulations are done for 7 prominent loess sites across the \acrshort{clp} and two-particle sizes to investigate how grain size is affected by the \acrshort{eawm} strength. 

The thesis is structured in the following way; \cref{seq:physics_of_dust} describe the role and impact of dust in the climate system which will motivate the discussion on
 

