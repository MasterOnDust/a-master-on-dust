\thispagestyle{plain}
\Chapter{}{Introduction}
% Need to introduce the word aeolian. 
The \acrfull{clp} is the largest and thickest deposit of cenozoic windblown dust. The \acrshort{clp} convers an area of approximately \SI{275600}{\kilo\metre\squared} and is more than 200\si{\metre} at its thickest. Dust from the proximal source regions has been naturally deposited over this region for millennials, faithfully recording the past environmental history of East Asia extending back 22Ma. The stratigraphy of the Quaternary \acrshort{clp} deposits is composed of layers of alternating loess and palaeosol sequences. The palaeosol sequences are linked to the interglacial periods during which East Asia experienced a more humid climate with  increased soil formation connected to a stronger East Asian summer monsoon. The loess sequences are though to be formed during the glacial periods which are thought to dominated by a stronger winter monsoon resulting in increased aridity and dust deposition. Thus the loess serves an important record of Quaternary glacial and interglacial periods in East Asian.

In reconstruction studies of the \acrfull{eawm} changes in 
\acrfull{mar} and loess grain-size are typically attributed to variations in the \acrshort{eawm} strength \parencite{stevens2007reinterpreting}. Emerging from these studies has the general claim that there is a strengthening of the EAWM during glacial stages causing an increase in \acrshort{mar} with a shift towards larger grain-sizes and conversely a significant downturn of EAWM activity during interglacial stages. However recently this interpretation has been cast under doubt based on fact that modern dust emissions in East Asia are primarily occurring in spring. Moreover the possible mechanisms linking \acrshort{eawm} to spring time emissions still remain an open question. \parencite{roe2009interpretation}. \textcite{wyrwoll2016cold} investigated the linkage between modern dust emissions and the \acrshort{eawm} by examining the inter annual correlations between visibility observations from several meteorological stations in china and the frequency of cold wave outbreaks over a 40 year period. In their study the cold wave frequency was used as a proxy for the \acrshort{eawm} strength. The cold wave frequency was found to be strongly correlated to the dust storm frequency and conversely \textcite{wyrwoll2016cold} concluded that \acrshort{clp} serves as valid record for the variability of \acrshort{eawm} strength. However the loess is formed through dust deposition, whereas the visibility does not necessarily reflect the amount of deposited dust, for instance during wet deposition episodes. Moreover \textcite{wyrwoll2016cold} only considered the frequency of the dust events, but it is possible that single strong dust events contribute majority of the dust deposited. Thus the motivation behind this thesis is to remedy the weaknesses of \textcite{wyrwoll2016cold} and examine whether the \acrshort{mar} is really linked to \acrshort{eawm} and if the grain-size would be a better indicator of \acrshort{eawm} strength. 
In this thesis, the \acrfull{lpdm} FLEXPART \parencite{Flexpart10.4_ref} and dust emission model FLEXDUST \parencite{flexdust_ref_2016} are employed for the first time to simulate desert dust mobilisation, transport and deposition in East Asia \acrshort{clp}. In contrast to conventional integrated dust models this model setup can be run backwards in time following the dust from receptor site to the source regions that makes it possible to establish a detailed map of how much each source element contribute to the deposition. The inter-annual variations in spring time dust emissions, transport and deposition over \acrshort{clp} are examined over a 20 year period during 1999-2019. The backward FLEXPART simulations are done for 7 prominent loess sites across the \acrshort{clp} and simulations are done for two particles sizes in order to investigate how grain size is affected by the \acrshort{eawm} strength. 

The thesis in structured in the following way; \cref{seq:physics_of_dust} describe the role and impact of dust in the climate system which will motivate the discussion on
 

