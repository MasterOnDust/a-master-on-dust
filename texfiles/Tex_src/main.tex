% REMEMBER TO SET LANGUAGE!
\documentclass[a4paper,12pt,english]{report}
\usepackage[utf8]{inputenc}
\usepackage{babel,duomasterforside}

\usepackage{master_on_dust}
% Alter some LaTeX defaults for better treatment of figures:
% See p.105 of "TeX Unbound" for suggested values.
% See pp. 199-200 of Lamport's "LaTeX" book for details.
%   General parameters, for ALL pages:
% \renewcommand{\topfraction}{0.9}	% max fraction of floats at top
% \renewcommand{\bottomfraction}{0.8}	% max fraction of floats at bottom
%   Parameters for TEXT pages (not float pages):
\setcounter{topnumber}{2}
\setcounter{bottomnumber}{2}
\setcounter{totalnumber}{4}     % 2 may work better
\setcounter{dbltopnumber}{2}    % for 2-column pages
\renewcommand{\dbltopfraction}{0.9}	% fit big float above 2-col. text
\renewcommand{\textfraction}{0.07}	% allow minimal text w. figs
%   Parameters for FLOAT pages (not text pages):
\renewcommand{\floatpagefraction}{0.7}	% require fuller float pages
% N.B.: floatpagefraction MUST be less than topfraction !!
\renewcommand{\dblfloatpagefraction}{0.7}	% require fuller float pages
% remember to use [htp] or [htpb] for placement

%Macro for writing vectors
\def\*#1{\ensuremath{\mathbf{#1}}}

\newcommand{\uveci}{{\bm{\hat{\textnormal{\bfseries\i}}}}}
\newcommand{\uvecj}{{\bm{\hat{\textnormal{\bfseries\j}}}}}
\DeclareRobustCommand{\uvec}[1]{{%
  \ifcat\relax\noexpand#1%
    % it should be a Greek letter
    \bm{\hat{#1}}%
  \else
    \ifcsname uvec#1\endcsname
      \csname uvec#1\endcsname
    \else
      \bm{\hat{\mathbf{#1}}}%
     \fi
   \fi
}}
\newcommand{\bvec}[1]{\mathbf{#1}}
%Fixing some ln shit!
\newcommand{\lnb}[1]{%
  \ln\mleft(#1\mright)%
}

% Macros for writing auto-sized paranthesis, brackets and braces
\newcommand{\para}[1]{\left(#1\right)}
\newcommand{\brak}[1]{\left[#1\right]}
\newcommand{\brac}[1]{\left\{#1\right\}}



\newcommand{\pd}[2][]{\frac{\partial#1}{\partial#2}}
%Listings configuration
\usepackage{listings}
%Hvis du bruker noe annet enn python, endre det her for å få riktig highlighting.
\lstset{
	backgroundcolor=\color{lbcolor},
	tabsize=4,
	rulecolor=,
	language=python,
        basicstyle=\scriptsize,
        upquote=true,
        aboveskip={1.5\baselineskip},
        columns=fixed,
	numbers=left,
        showstringspaces=false,
        extendedchars=true,
        breaklines=true,
        prebreak = \raisebox{0ex}[0ex][0ex]{\ensuremath{\hookleftarrow}},
        frame=single,
        showtabs=false,
        showspaces=false,
        showstringspaces=false,
        identifierstyle=\ttfamily,
        keywordstyle=\color[rgb]{0,0,1},
        commentstyle=\color[rgb]{0.133,0.545,0.133},
        stringstyle=\color[rgb]{0.627,0.126,0.941}
        }




% \setlength{\parindent}{1em}

\title{My master thesis}
\subtitle{To years of work}
\author{Ove Haugvaldstad}

\begin{document}
\duoforside[program={Meteorology and Oceanography},
  dept={Section of Meteorology and Oceanography \and Department of Geoscience},
  long]
  
\tableofcontents

\newpage
\chapter{Background}
Aeolian dust is a fundamental component in the climate system. Large range of scales motion influences the emission 
,transport and deposition of aeolian dust from turbulent wind gust at the surface to large scale atmospheric circulation 
patterns. The aim of this chapter is to the provide the necessary background for understanding the emission, deposition 
and transport of East Asian aeolian dust. The first part of this chapter will focus on explaining the physics of dust 
emissions, what are different processes and how do we model dust emissions, where are the major mineral dust source in 
East Asia and which kind of meteorological conditions favours dust emission. The second part will go into the transport 
of aeolian dust, the role large scale circulation have on dust transport from East Asian desert. \todo{NEED to give 
proper references to the different section}. The last part will look at the how mineral dust is deposited, how was the 
Loess Plateau formed and how the Loess deposit can be an a window into understanding past climate.   
\subfile{../Background/physics_of_windblown_dust}
\subfile{../Background/Mineral_dust_impact}
\chapter{Methods}
\subfile{../Methods/flexdust_model_discription}
\subfile{../Methods/flexpart_model_discription}
\newpage
\printbibliography

\end{document}