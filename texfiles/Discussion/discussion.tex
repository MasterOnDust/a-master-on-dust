\Chapter{}{Discussion}\label{chap:Discussion}
  
\section{TODO:}

\begin{itemize}
    \item Define research question
    \item I haven't been there
\end{itemize}

\section{The spatial difference in the sources of dust deposition over CLP}
The FLEXPART/FLEXDUST simulations indicated the North West \acrshort{clp} as the main source region for the dust deposited at the sites included in this study. Spapotou, Yinchuan and SACOL, which are all located on the western side of the Liupan Mountains, has a large contribution from the Tengger and Badain Jaran desert as opposed to Badoe, Luochuan and Lantian, which are more influence by the deserts located on the Inner Mongolia Plateau. The topography could explain the difference in source regions between SACOL and Lingtai, where SACOL experience less dust transport from the more northerly deserts and vice versa for Lantian. This interpretation is also consistent with the centroid dust loading trajectories, which are in a more easterly direction compared to the other locations. 

The north south topography of the \acrshort{clp} might explain in the difference in the amount of deposition at Luochuan and Lantian. Luochuan and Lantian have very similar source regions and the amount of dry deposited material (\Cref{fig:source_contrib_2mmu}). However, Luochuan experience much less wet deposition. The difference in elevation between Lantian and Luochuan is around 500 meters. As moisture is typically transported from the south, a significant portion of the precipitation might have precipitated before arriving at Luochuan. In fact, the precipitation at the ERA5 grid cell nearest to Luochuan is about 30\% less than Lantian. In addition, the wet deposition centroid dust loading trajectories for the two sites are nearly identical, suggesting that the same frontal systems are causing precipitation at both locations.              


%  Which is consistent previous modelling studies \textcite{shi2011distinguishing}.

\section{Transport of fine and coarse dust to the \acrshort{clp}}
The model result suggested a stronger contribution from remote sources for the site located at the western part of the \acrshort{clp} compared to the eastern sites. This spatial difference was largest for the silt particles. The   
Opposed to the silt particles the clay particles could be transported within the lower troposphere even from the most remote sites (\Cref{fig:dust_loading_trajecs}).  For the silt particles to be able to reach the \acrshort{clp} from the remote source required to be entrained much higher into the atmosphere. There are three possible mechanisms that could carry the dust into the upper troposphere, (1) turbulent eddies, (2) convection and (3) mechanically forced lifting. The two first mechanisms predominantly dependent on the atmospheric stability. In the Tarim Basin \textcite{yumimoto_elevated_2009} demonstrated slopes of the \acrfull{ntp} could produce updraft winds strong enough to lift the dust veil to height between 9-12 km. Comparisons of Zircon U-Pb geochronologies from the \acrshort{clp} and the surrounding source regions have revealed Taklamakan, Western Mu Us and Northern Tibetan Plateau as the as the most likely source for loess at the \acrshort{clp} \parencite{bird2015quaternary}. Due to it's distance from the depositioncentre and the dominant medium to coarse silt composition of the loess, Taklamakan is often disregarded as a major source to the \acrshort{clp}.                 





What are the causes for the spatial difference


\section{Mechanism for the interannual variation of dust deposition over CLP }
The polar cold air outbreaks are well known to be the driving factor of dust production in North East Asia. The intensity of the winter monsoon is often thought to regulate the cold air outbreak activity, the analysis of interannual variation in dust deposition over \acrshort{clp} pointed out a very weak correlation between \acrshort{eawm} strength and  


\subsection{emission?}



\subsection{transport?}
\subsection{fine vs. coarse, wet vs. dry?}
\subsection{spatial difference in interannual variation?}
\subsection{Link between AO and dust deposition over CLP?}
The winter circulation composite anomalies and the correlation analysis showed a strong connection between the winter AO and dust deposition. During negative AO years the jetstream is weaker causing it to meander further south. A weaker yet stream does also make it more favourable for cold air intruding into East Asia \parencite{he2017impact}. Negative \acrshort{ao} is also associated with a increased snow and cloud cover over central Siberia which increases the albedo amplifying the surface cooling.  \textcite{liu2018influence} examined the relationship between spring dust emissions and preceding winter \acrshort{ao} using the MERRA2 reanalysis showed that the anomalously cold surface conditions over Central Siberia that arise during negative \acrshort{ao} years can produce anomalously cold conditions that last into spring. Moreover the cold conditions strengthens the temperature gradient between Mongolia and northern China causing stronger than usual north westerlies. This is consistent with the spring composite anomalies of EOF1 in \Cref{fig:eof_composite_MAM}a which shows anomalous cyclonic circulation over Central Siberian and strengthened North westerlies. However from the spring time composite anomalies for the total deposition  \Cref{fig:MAM_850_fine_composite} and \Cref{fig:MAM_850_coarse_composite} only Yinchuan similar circulation anomalies to the emissions. 

Comparing the spring EOF2 circulation composite anomalies with the deposition circulation composite anomalies, SACOL fine clay \Cref{fig:MAM_850_fine_composite}e show similar circulation anomalies to that of positive EOF2 years. Accordingly strong deposition of fine dust at SACOL suggests a large contribution from Taklamakan, this is also consistent with source contribution composite anomalies in \Cref{fig:source_contrib2mmu_anomalies}. Interestingly Luochuan coarse silt show circulation anomalies more in line with the negative EOF2 years.    

\section{Implication for interpreting loess records}


\section{FLEXPART as tool for studying dust transport} 
\subsection{Trajectory analysis in FLEXPART}
The 3D trajectories of the dust plume are important for understanding how the dust are transported in the atmosphere. However since FLEXPART calculates the trajectories for thousands of particles, looking at the individual particle trajectories would require processing a lot of information. A method to condense the trajectories is to do clustering of the particle positions. Clustering routines already exist in FLEXPART, but it's value is limited by only clustering the particle positions at each time step. Thus the cluster trajectories produced from the clustering routine does not give a continuous trajectory of the actual cluster centriods of the particle plume, because which cluster it previously belonged is not taken into consideration. Due to this flaw in the clustering routine it was not possible to advantage of the clusters produced by FLEXPART for analysing the dust transport trajectories. Rather the only meaningful approach was to use the plume centriod trajectories, but the caveat of this approach is that the centriod is usually not representative of the dust transporting trajectories, since it is an average of both dust and none dust carrying particle trajectories.

Improving the clustering routine in FLEXPART would involve doing clustering based on the trajectory as a whole. This could be achieved by reading in the trajectory file at each time step and then assign the clustered points to the closest cluster centriod. The most accurate approach would be to store the trajectory of each particle either in a file or in memory and then re-cluster the trajectories at each time step. However the performance penalty of this approach might be to large.

\subsection{FLEXPART output format}

I would argue that the current format of the backward emissions sensitivity is not optimal. As described in \Cref{sec:flexpart} the deposition and source contribution is obtained by multiplying the emissions sensitivity with the emission strength. However the current output format has the dimmensionallity of (pointspec, time, height, lon, lat) where the time dimension starts and the end of the simulation a  then goes backward to the beginning. The pointspec dimension represent the forward time dimension which corresponding to a new particle release. The length of the backward trajectories are then controlled by specifying the maximum age through the AGECLASS parameter. First of all the current format result in a lot of empty entries as when the maximum age is exceeded. In therms of the ease of use of the output from backward FLEXPART simulations the output format is a major problem and can easily cause miss understandings. The current output format is designed to be same regardless of the whether it done using a forwad or backward approach. However I belive that FLEXPART would be easier to use if an output format specifically for the backward simulation were to be included. Which had the dimensions of (btime, time, height, lon,lat) where btime correpsond to the time along the back trajectory in seconds and time would be the forward time dimensions. Then the output from FLEXPART could be use directly without having to post processes first. 
