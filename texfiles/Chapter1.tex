\thispagestyle{plain}
\chapter{Introduction}
\section{Introduction}
% \addcontentsline{toc}{chapter}{Introduction}
% Need to introduce the word aeolian. 
The windblown sediments at the \acrfull{clp} compromise the largest and thickest deposit of Cenozoic windblown dust on Earth.  The \acrshort{clp} covers an area of approximately \SI{450000}{\kilo\metre\squared} and is at the thickest, more than 350\si{\metre} thick \parencite{liu1985loess}.
Dust from the proximal source regions has been naturally deposited over this region for millennials and faithfully recorded the past environmental history of East Asia. 
The aeolian dust deposits on the \acrshort{clp} extends back 22Ma \parencite{qiang2011new} and in adjecent areas even as far back as 35Ma \parencite{licht2014asian,wasiljeff2020magnetostratigraphic}. 
The stratigraphy of the Quaternary \acrshort{clp} deposits is composed of alternating loess and palaeosol units \parencite{derbyshire1995variations}. 
The palaeosols are linked to the interglacial periods during which East Asia experienced a more humid climate with increased soil formation connected to a stronger East Asian summer monsoon \parencite{maher1992paleoclimatic}. 
The loess units deposited during the glacial periods and are considered to be formed during enhanced winter monsoon conditions resulting in increased aridity and increased dust deposition \parencite{xiao1995grain}.
Thus the Chinese loess serves as an important record of Quaternary glacial and interglacial periods in East Asia \parencite{kohfeld2003glacial,an2014cenozoicChange}.
The older Red Clay sequence situated underneath the Quaternary sequence and is of a similar aeolian origin, however, the Red Clays more contineous without interluding palesol units and is better pedogenically developed. Still, the source regions and transport processes (predominately by westerlies or winter monsoon winds) of Red Clay and Quaternary loess are still debated \parencite{shang2016variations,bird2015quaternary,miao2004spatial}.

Changes in \acrfull{mar} and loess grain size have typically been attributed to variations in the \acrshort{eawm} strength \parencite{stevens2007reinterpreting}.
Emerging from these studies has the general claim that there is a strengthening of the \acrshort{eawm} during glacial stages, causing an increase in \acrshort{mar} over \acrshort{clp} with a shift towards larger grain sizes.
Conversely, the humid interglacial periods are considered periods of significant downturn of \acrshort{eawm} activity.
% Thus  the interglacial  periods  are characterised by a  
However, this interpretation has been cast under doubt because modern dust emissions and deposition in East Asia are primarily a spring phenomenon \parencite{sun2001spatial}.
Moreover, the possible mechanisms linking the \acrshort{eawm} to springtime emissions remains an open question \parencite{roe2009interpretation}. 
In the study by \citeauthor{wyrwoll2016cold}, the linkage between modern dust stroms and the \acrshort{eawm} was investigated by examining the interannual correlations between visibility observations from several meteorological stations across China (a indicator of dust storm) and the frequency of cold wave outbreaks over a 40 year period. 
In their study, the cold wave frequency was used as a proxy for the \acrshort{eawm} strength. 
The cold wave frequency was found to be strongly correlated with the dust storm frequency in spring.
Consequently, \citeauthor{wyrwoll2016cold} concluded that the \acrshort{clp} serves as a valid record for the variability in the \acrshort{eawm} strength.
However, \citeauthor{wyrwoll2016cold} does not consider that visibility is primarily an indicator of the atmospheric dust concentration rather than dust deposition. 
Dust concentration and deposition are not necessarily coupled, for instance during wet deposition episodes, which are influenced by other factors in addition to the mean flow \parencite{osada2014wet}. 
% As wet deposition is influenced by other factors in addition to the mean flow.
Moreover, they only consider the frequency of dust events, however single strong dust events may contribute to the majority of the dust deposited \parencite{ta2004measurements}. 
Thus, the motivation behind this thesis is to address the shortcommings of \textcite{wyrwoll2016cold} and examine whether the \acrshort{mar} (i.e.,) over the \acrshort{clp} is really linked to \acrshort{eawm} and why.

In this thesis, the state-of-the-art \acrfull{lpdm}  \acrfull{flexpart} \parencite{Flexpart10.4_ref} and the dust emission model FLEXDUST \parencite{flexdust_ref_2016} are employed for the first time to simulate \acrfull{eadc}.
Whereby dust cycle I refer to the fast processes of dust cycle namely; dust emission from the source regions, dust transport in the atmosphere, and dust removal by wet or dry deposition. Specifically, I will investigate the dust cycle of the dust deposited over the \acrshort{clp}, referred to as the \acrshort{eadc}.
In contrast to  models previously used to study the \acrshort{eadc} \parencite{gong2006simulated,shi2011distinguishing,liu2018influence}, this model setup is run backwards in time, following the dust from the receptor back to the source. This setup makes it possible to establish a detailed map of how much each source contributes to the deposition at specific location. 
The backward FLEXPART simulations are done for 7 prominent loess sites across the \acrshort{clp} and two-particle sizes are used to investigate how grain size of the deposited dust is affected by the \acrshort{eawm} strength. All the analyses of the \acrshort{eadc} over \acrshort{clp} are based on 20 year simulation of \acrshort{eadc} during spring from 1999 to 2019.

This work is a part of a larger research project Tracing the Winds, funded by the Academy of Finland and coordinated by Anu Kaakinen. The main aim of this project is to obtain a comprehensive understanding of the dust signal in the Neogene Red Clay deposits in order to better constrain their interpretation as palaeoclimate proxies. The idea is that a good understanding of the present-day spatio-temporal variation of dust cycle can help better interpretating the long-term variation of the \acrshort{eadc} on time scales from millenials to million of years as revealed by the red clay and loess records.

\section{Outline and research questions}
The thesis is structured according to three main themes; (1) Model assessment, (2) Spatial pattern of the \acrshort{eadc} and the dust source of the loess plateau and (3) Interannual variation of the \acrshort{eadc} and the source of the loess plateau. Each of the components of the \acrshort{eadc}, emission, transport and deposition will be examined from the perspective of these overarching themes for the selected loess site at the \acrshort{clp}. Moreover, the discussion and analysis of each theme is guided by its own set of research questions: 
\begin{itemize}
    \item \textbf{Model assessment;} How well does FLEXPART and FLEXDUST model represent the \acrshort{eadc}, in particular, dust emission, and dry and wet deposition at a location? What are the possible uncertainties and biases? How does the model uncertainties impact the interpretation of the model results? What are the advantages/disadvantages of this modelling approach?
    \item \textbf{Climatology of the \acrshort{eadc};} Is wet or dry deposition the main mode of deposition to the \acrshort{clp}? How do the source areas and dust transport paths among the sites over the loess plateau differ? Are the source areas different for the coarse (dry) and fine (wet) deposited dust? 
    \item \textbf{Interannual variation of the \acrshort{eadc};} What are the differences in circulation patterns, dust transport paths and source regions between strong and weak deposition years? How does the interannual variation of winter circulation affect springtime dust cycle? 
\end{itemize}
The insights from each theme will then be synthesised to discuss the possible implications for the interpretation of the Chinese Red Clays and Loess records. 

The remaining sections are organized in the following way: 
% Section 2 can be "Theory"
\Cref{chap:dust_roles_models} establishes dust as an important component in the climate system and describe the theory related to dust modelling. 
\Cref{chap:east_asia_dust} describes the East Asian dust sources and the \acrshort{clp}. 
The FLEXDUST and FLEXPART models are described in the \Cref{chap:methods}, in addition to input data, model set up and analysis methods. The results are presented in \Cref{Chap:Results} following the three themes of this thesis. 
The results are then discussed in \Cref{chap:Discussion}, the discussion is guided by the research questions defined above. Finally, \Cref{chap:conclusion} will conclude the thesis by giving a summary of the main findings and provide an outlook for future work. 
