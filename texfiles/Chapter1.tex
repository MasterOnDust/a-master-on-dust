\thispagestyle{plain}
\chapter{Introduction}
% \addcontentsline{toc}{chapter}{Introduction}
% Need to introduce the word aeolian. 
The windblown sediments at the \acrfull{clp} compromise the largest and thickest deposit of Cenozoic windblown dust on Earth. The \acrshort{clp} covers an area of approximately \SI{275600}{\kilo\metre\squared} and is at the thickest, more than 200\si{\metre} thick \todo{reference}.
Dust from the proximal source regions has been naturally deposited over this region for millennials and faithfully recorded the past environmental history of East Asia. 
The oldest parts of the Chinese loess sequences have been dated back to 22Ma \parencite{qiang2011new}. 
The stratigraphy of the Quaternary \acrshort{clp} deposits is composed of layers of alternating loess and palaeosol sequences \parencite{derbyshire1995variations}. 
The palaeosol sequences are linked to the interglacial periods during which East Asia experienced a more humid climate with increased soil formation connected to a stronger East Asian summer monsoon\textbf{reference}. 
The loess sequences formed during the glacial periods, which are thought to have been dominated by a stronger winter monsoon resulting in increased aridity and dust deposition\textbf{reference}.
Thus the Chinese loess serves as an important record of Quaternary glacial and interglacial periods in East Asia \textbf{reference}.

In reconstruction studies of the \acrfull{eawm}, changes in \acrfull{mar} and loess grain size are typically attributed to variations in the \acrshort{eawm} strength \parencite{stevens2007reinterpreting}.
Emerging from these studies has the general claim that there is a strengthening of the \acrshort{eawm} during glacial stages, causing an increase in \acrshort{mar} over \acrshort{clp} with a shift towards larger grain sizes. Thus  the interglacial  periods  are characterised by a  significant downturn of \acrshort{eawm} activity.
However, this interpretation has been cast under doubt because modern dust emissions in East Asia is primarily a spring phenomenon \parencite{sun2001spatial}.
Moreover, the possible mechanisms linking the \acrshort{eawm} to springtime emissions remains an open question \parencite{roe2009interpretation}. 
\textcite{wyrwoll2016cold} investigated the linkage between modern dust emissions and the \acrshort{eawm} by examining the interannual correlations between visibility observations from several meteorological stations across China and the frequency of cold wave outbreaks over a 40 year period. 
In their study, the cold wave frequency was used as a proxy for the \acrshort{eawm} strength. 
The cold wave frequency was found to be strongly correlated to the dust storm frequency.
Conversely, \textcite{wyrwoll2016cold} concluded that the \acrshort{clp} serves as a valid record for the variability in the \acrshort{eawm} strength.
However, the loess is formed by dust deposition, and the visibility is primarily an indicator of the atmospheric dust concentration. 
Dust concentration and deposition are not necessarily coupled, for instance during wet deposition episodes\textbf{reference}. 
Moreover, \textcite{wyrwoll2016cold} only considered the frequency of the dust events, however single strong dust events may contribute to the majority of the dust deposited. 
Thus the motivation behind this thesis is to address the weaknesses of \textcite{wyrwoll2016cold} and examine whether the \acrshort{mar} over the \acrshort{clp} is really linked to \acrshort{eawm}.

In this thesis, \acrfull{flexpart} \parencite{Flexpart10.4_ref} and dust emission model FLEXDUST \parencite{flexdust_ref_2016} are employed for the first time to simulate \acrfull{eadc}. Whereby dust cycle I refer to the fast processes of dust cycle namely; dust emission from the source regions, dust transport in the atmosphere, and dust removal by wet or dry deposition. Specifically, I will investigate the dust cycle of the dust deposited over the \acrshort{clp}, referred to as the \acrshort{eadc}.
In contrast to a integrated dust model, this model setup is run backwards in time, following the dust from the receptor back to the source. This setup makes it possible to establish a detailed map of how much each source contributes to the deposition. 

Both the climatology and interannual variability of the \acrshort{eadc}  over \acrshort{clp} are based on 20 year simulation of \acrshort{eadc} during spring from 1999 to 2019.  
The backward FLEXPART simulations are done for 7 prominent loess sites across the \acrshort{clp} and two-particle sizes are used to investigate how grain size of the deposited dust is affected by the \acrshort{eawm} strength. 

\section{Outline and research questions}
The thesis is structured according to three main themes; (1) Model assessment, (2) Climatology of the \acrshort{eadc} and (3) Interannual variability of the \acrshort{eadc}. Each of the components of the \acrshort{eadc}, emission, transport and deposition will be examined from the perspective of these overarching themes. Moreover, the discussion and analysis of each theme is guided by its own set of research questions: 
\begin{itemize}
    \item \textbf{Model assessment;} How well does FLEXPART and FLEXDUST model represent the \acrshort{eadc}? What are the possible uncertainties and biases? How does the model uncertainties impact the interpretation of the model results? What are the advantages/disadvantages of this modelling approach?
    \item \textbf{Climatology of the \acrshort{eadc};} What is the main mode of deposition to the \acrshort{clp}? How are the source areas and dust transport paths among the sites different? Is the source areas different for the coarse and fine/wet and dry deposited dust? 
    \item \textbf{Interannual variability of the \acrshort{eadc};} What are the differences in circulation patterns, dust transport paths and source regions between strong and weak deposition years? How does the interannual variability of winter circulation affect springtime dust? 
\end{itemize}
The insights from each theme will then be synthesised to discuss the possible implications for the interpretation of the Chinese Loess records. 

The thesis is structured in the following way: 
\Cref{chap:dust_roles_models} establishes dust as an important component in the climate system and describe the theory related to dust modelling. 
\Cref{chap:east_asia_dust} describes the East Asian dust sources and the \acrshort{clp}. 
The FLEXDUST and FLEXPART models are described in the \Cref{chap:methods}, in addition to input data, model set up and analysis methods. The results are presented in \Cref{Chap:Results} following the three themes of this thesis. 
The results are then discussed in \Cref{chap:Discussion}, the discussion is guided by the research questions defined above. Finally, \Cref{chap:conclusion} will conclude the thesis by giving a summary of the main findings and provide possibilities for future work. 
