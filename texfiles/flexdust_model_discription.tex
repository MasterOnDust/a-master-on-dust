\subsection{FLEXDUST}
To estimate the dust emission flux, the emission module FLEXDUST was used   
\parencite{flexdust_ref_2016}. The FLEXDUST is a standalone emission module
created for FLEXPART, the dust flux estimation relies on simple formulations
common in global dust transport models. 

FLEXDUST identifies possible mineral dust source from the bare soil fraction.
Here the bare soil fraction is obtained from The Global Land Cover by National
Mapping Organizations (GLCNMO) version 3 \parencite{shirahata2017production}
dataset, which based on remotely sensed MODIS (or Moderate Resolution Imaging
Spectroradiometer) data carried onboard the Terra and Aqua earth observing
satellites. Both bare soil and partly vegetated areas are considered as possible
dust sources. The available soil fraction for vegetated areas is determined by
subtracting the vegetation cover used by as used ECMWF. FLEXDUST also takes into
consideration that sediments are more easily gathered in depressions,
consequently these areas should be more favorable for dust emission
\parencite{zender2003mineral}. Defining the erodible soil fraction
(\cref{eq_ero_soil_frac}) according to \textcite{dust_dist_Ginoux2001}.

\begin{equation}\label{eq_ero_soil_frac}
    S = \left(\frac{z_{max} - z_i}{z_{max} - z_{min}}\right)^5 
\end{equation}    

Here $z_i$ is the local elevation and $z_max$ and $z_min$ is the minimum and
maximum elevation in a 10\degree $\times$ 10\degree area. The erodibility $S$ is
then scaled by the bare soil fraction to get the erodible soil fraction in (\textbf{Include fig}). 

Due to cohesive force between the soil properties the majority of windblown dust
is not aerodynamically lifted into the atmosphere. Rather they are typically
ejected by impacting dust particles, in a process known as saltation. If the energy impact of the 
impacting dust particle is large enough damaging or fracturing the particle at rest producing, ejecting many particles into the atmosphere. 
 