\subsection{FLEXDUST}
To estimate the dust emission flux, the emission module FLEXDUST was used   
\parencite{flexdust_ref_2016}. The FLEXDUST is a standalone emission module
created for FLEXPART, the dust flux estimation relies on simple formulations
common in global dust transport models. 

To identify possible mineral dust sources FLEXDUST require the bare soil
fraction. Here the bare soil fraction are obtained from The Global Land Cover by National Mapping
Organizations (GLCNMO) version 3 \parencite{shirahata2017production} dataset, which
based on remote sensing data from MODIS (or Moderate Resolution Imaging
Spectroradiometer) carried onboard the Terra and Aqua earth observing satellite.
Both bare soil and partly vegetated areas are considered as possible dust
sources. The available soil fraction for vegetated areas is determine by
subtracting the vegetation cover used by ECMWF. FLEXDUST also takes into
consideration that sediments are more easily gathered in depressions, which
means that these areas should be more favorable for dust emission
\parencite{zender2003mineral}. Defining an erodible soil fraction (\cref{eq_ero_soil_frac}) according to \textcite{dust_dist_Ginoux2001}.
\begin{equation}\label{eq_ero_soil_frac}
    S = \left(\frac{z_{max} - z_i}{z_{max} - z_{min}}\right)^5 
\end{equation}    
Here $z_i$ is the local elevation and $z_max$ and $z_min$ is the minimum and
maximum elevation in a 10\degree $\times$ 10\degree area. The erodibility $S$ is
then scaled by the bare soil fraction to get the erodible soil fraction in (\textbf{Include fig}). 

Soil properties is also have a large influence on the dust emission. In the  