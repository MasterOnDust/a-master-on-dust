
\Chapter{}{Discussion}\label{chap:Discussion}

Here I will introduce what I am going to discuss

% The consequences of the low agreement between model and observation on the  wet deposition flux  and other model uncertainties for the interpretation and analysis of the model results will be discussed in \Cref{sec:model_uncertainties}. 
% The results presented in the last chapter also showed spatial differences in the multiyear mean transport trajectories and source contribution for the different receptor locations. 
% These differences will be discussed further in \Cref{sec:spatial_differences}. 
% Moreover, the results also showed a disproportionally large contribution of wet deposited  dust to the total deposition at all the sites and particle size. 
% The implication of this result will be discussed in \Cref{sec:wetdep_discussion}. 
% The results suggested a strong influence from the preceding winter \acrshort{ao} on dust emissions and deposition. 
% However, the more striking result was the weak influence of the \acrshort{eawm} strength on the deposition in disagreement with the conclusion of \textcite{wyrwoll2016cold}.
% Therefore, the linkage between winter circulation and springtime deposition will be discussed in the \Cref{sec:linkage_winter_dust}. 
% I will end by highlighting the implications of these results for the interpretation of the loess record.  

% The previous chapter the results from the FLEXPART and FLEXDUST simulations. When compared to observations FLEXPART/FLEXDUST performed well, especially for the dry deposition. However, the wet deposition did not compare that well to the observations. Conversely, the uncertainties related to the wet deposition, and how this would impact the results will be discussed first in \Cref{sec:model_uncertainties}.
% and  model setup was shown to be able to represent the 
%   Start with a few sentences that summarize the most important results.
% What are the major patterns in the observations? (Refer to spatial and temporal variations.)
% What are the relationships, trends and generalizations among the results?
% What are the exceptions to these patterns or generalizations?
% What are the likely causes (mechanisms) underlying these patterns resulting predictions?
% Is there agreement or disagreement with previous work?
% Interpret results in terms of background laid out in the introduction - what is the relationship of the present results to the original question?
% What is the implication of the present results for other unanswered questions in earth sciences, ecology, environmental policy, etc....?
% Multiple hypotheses: There are usually several possible explanations for results. Be careful to consider all of these rather than simply pushing your favorite one. If you can eliminate all but one, that is great, but often that is not possible with the data in hand. In that case you should give even treatment to the remaining possibilities, and try to indicate ways in which future work may lead to their discrimination.
% Avoid bandwagons: A special case of the above. Avoid jumping a currently fashionable point of view unless your results really do strongly support them. 
% What are the things we now know or understand that we didn't know or understand before the present work?
% Include the evidence or line of reasoning supporting each interpretation.
% What is the significance of the present results: why should we care? 

\section{Model assessment}

An important objective of this work was to determine whether \acrshort{flexpart} and FLEXDUST could be used to model the \acrshort{eadc}. The model validation showed that \acrshort{flexpart} could reasonably reproduce the interannual and spatial variability of dust deposition when compared to observations. 
However, there were large discrepancies between the model and observations on the monthly amount of wet deposition. 
The wet deposition bias will be discussed in \Cref{sec:model_uncertainties}. 
Sensitivity experiments were done to identify how possible differences and errors in the input data would affect the model performance. 
It was shown that the choice of soil texture data could greatly impact the estimated dust emissions. 
In addition, forcing sensitivity experiments showed that the FLEXDUST produced much less dust emissions when forced using ERA-Interim than ERA5. 
The difference was particularly large over the Taklamakan. 
The differences between the two forcing data sets were most noticeable in the source contribution map.
The implications of the sensitivity experiments will be discussed in \Cref{sec:sensitivity_experiments_implications}
This work has showcased the capabilities and advantages of this modelling setup. 
The information obtained from the source contribution made it possible to examine the spatial variability in source regions to \acrshort{clp} with more details than previously possible. 
However, this model setup is not without its flaws. 
Therefore, based on my experience working with FLEXPART and FLEXDUST, I will discuss the main advantages and disadvantages of this modelling set up  \Cref{sec:model_advantages}. I will conclude the model assessment by giving some for further improvement of the modelling setup.

\subsection{Wet deposition bias}\label{sec:model_uncertainties}
Uncertainties and model limitations are embedded in any modelling study, and considering these uncertainties is an important piece of interpreting the results.  
The sensitivity experiments and model evaluation highlighted some of the limitations of FLEXPART/FLEXDUST. 
In particular, the inability of FLEXPART to reproduce the variability in the wet deposition flux compared to observations. 
Achieving a good performance of the wet deposition scheme is particularly important, given that based on the model simulation wet deposition is the primary mode of deposition to the \acrshort{clp}. 
\textcite{flexdust_ref_2016} achieved relatively good agreement between FLEXDUST combined with a forward FLEXPART simulations and observations of annual dust deposition fluxes using 10 size bins. 
A possible reason for why \textcite{flexdust_ref_2016} achieved good agreement with observations is that they compared annual deposition fluxes, and it is possible and that the apparent biases on the monthly level compensated each other such that the total deposition was around the observed value. 
Besides \textcite{flexdust_ref_2016}, there have not been other studies that have evaluated the performance of FLEXPART to simulate deposition of mineral dust.
However, there have been more studies done comparing simulated deposition of black carbon to with observations, e.g. \textcite{flexpart_wetdep, choi_investigation_2020}. 
\textcite{choi_investigation_2020} suggested that FLEXPART was not able to capture the regional variability due to severely underestimating the in-cloud scavenging coefficient. Moreover, by diagnosing the relative importance of the input variables 
they found that the convective available potential energy (CAPE), should be considered an important parameter to improve the in-cloud scavenging scheme. 
Moreover, there is also a known issue with the current wet deposition scheme related to the interpolation and disaggregation of the input variables \parencite{tipka2021effects}. 
I did not have the time to test with the new interpolation scheme in FLEXPART. 
That would require preparing all the forcing data again, and support for the new interpolation scheme has not yet officially been released.

Another factor that disproportional impacts the estimated wet deposition is the spatial and temporal distribution of the dust emissions simulated in FLEXDUST. 
As shown from the input forcing sensitivity experiment, the accumulated source contribution showed a higher sensitivity to the input forcing than the FLEXDUST accumulated emission flux and FLEXPART emission sensitivity. Given how episodic the wet depositions events are and the dependence of emissions in the source region and precipitation at receptor location following the dust event. Slight changes in the temporal and spatial evolution of the dust emissions can greatly affect the amount of dust deposited at the receptor. Moreover, the analysis showed that the strong deposition events did not necessarily coincide with the strongest dust events. The agreement between ERA5 and ERA-Interim FLEXDUST simulations were higher for the strong dust events compared to the weaker dust events. Because the large synoptic scale dust events are better represented in the forcing data. The resulting bias towards strong dust events is less important for the dry deposition flux since the dry deposition strongly correlated with the dust emissions as shown in \Cref{fig:correlations}. 

Then how does the wet deposition bias affect the analysis?
An important factor that FLEXPART should get right is the interannual variability in the amount of springtime deposited dust to the \acrshort{clp}. 
As examining the interannual variability of the \acrshort{eadc} is an important part of this thesis. 
Based on the two years of data used for validation here, FLEXPART does agree with the observation on the year to year difference. 
However, from only two years of data it is hard to say that this is generally the case.
Moreover, since the total deposition is so heavily skewed towards wet deposition, the source contribution primarily depicts the source regions during the strong wet deposition episodes.  
Therefore, the wet deposition is a caveat of this study and has to be improved upon in the future. 

\subsection{Implications of model sensitivities}\label{sec:sensitivity_experiments_implications}
The model sensitivity reviled that the model forcing and  essential 

\subsection{Dust emissions and source contribution}
FLEXDUST estimated an average 11Mt of springtime dust emitted from the East Asian source regions. This estimate is lower than previous estimates by \textcite{xuan2004identification}. 
However, the important figure is not the total amount of dust emission. What is important, is whether FLEXDUST can capture the spatial distribution of the dust sources. \textcite{liu2018influence} compared the spatial distribution of dust emissions from the MERRA2 reanalysis to satellite-derived \acrfull{aod}. They found good agreement with spatial distribution of \acrshort{aod} and dust emissions. The spatial distribution of dust emissions in MERRA2 is very similar to FLEXDUST. Given that the spatial distribution of emissions is well represented in FLEXDUST in comparison with models, suggest that biases in modelled wet deposition are more likely caused by FLEXPART. 

Since this is the first modelling study of source attribution of the dust deposited over the \acrshort{clp} at such high resolution, only a few previous studies are relevant for comparison. \textcite{shi2011distinguishing} used a regional climate model to investigate the provenance of the dust deposited to the \acrshort{clp}. 
However, they only did source attribution for five source regions and therefore could not examine the inhomogeneities within each source region.
They found in agreement with FLEXPART/FLEXDUST that sources northwest of the \acrshort{clp} had the largest influence on the \acrshort{mar} over the \acrshort{clp}. 

\subsection{Strengths and weaknesses}\label{sec:model_advantages}
The ability of FLEXPART/FLEXDUST to directly establish which dust sources  are contributing to the dust deposition is what makes this modelling set up potent a tool examining the spatial variability in the dust sources to the \acrshort{clp}. Given a realistic representation of the dust sources, this modelling setup can be used to study the origin of loess sequences worldwide. 
% More generally given an emission inventory and realistic description of the tracer, FLEXPART can be used to map sources for any receptor.

Nonetheless, this modelling approach is not without flaws.
A major weakness of \acrshort{flexpart} is the complicated setup process.
Compared to the similar HYSPLIT model developed by NOAA \parencite{draxler2010hysplit}, which has a graphical interface and even the possibility to run simulations directly from the browser, the difference is huge. Despite HYSPLIT being less capable compared to FLEXPART in certain applications.
Still, due to how easy HYSPLIT is to use, the userbase of HYSPLIT is about six times that of FLEXPART \footnote{Based on the number of hits on google scholar between 2020 and 2021}. Thus, there is a clear advantage to having a model that is easy to use for the amount of science produced.   
Nevertheless, the difficulty and amount of effort required to set up and use a model is seldom given much attention. 
This modelling setup is further complicated by the lack of proper tools for post processing and analysis of the model output. 
For me, this meant that probably 50\% of the time spent working on this thesis have been spent on creating post analysis scripts for FLEXPART.
Moreover, there has not been a community effort to develop proper tools for analysing the FLEXPART output. 
The codes out there are mainly codes developed for some research project and after the end of the project is not maintained anymore.

Another deficiency is the lack of proper trajectory clustering capabilities. This made it impossible to diagnose the dust transport trajectories properly. 
Even though it is possible to do a proper cluster analysis in FLEXPART by saving the  individual trajectories and then do the clustering as a post-processing step. However, this easily gets quite computationally demanding. In addition, one has to store a lot of additional data, which increases the storage requirements.

\subsection{Suggestions for future improvement}\label{sec:conclusion_improvement}
This section has discussed the biases, sensitivities, advantages and disadvantages with the FLEXPART/FLEXDUST modelling setup. 
FLEXPART/FLEXDUST can become an indispensable tool for examining the dust sources and transport pathways involved in the formation of loess around the world. 
However, this work has shown that there are issues that have to be addressed first. 
Foremost, the wet deposition bias have to be addressed, a first step would be to run FLEXPART with new interpolation scheme. 
Moreover to allow for properly diagnosing the dust transport, the clustering routine in FLEXPART should be improved upon. Such that that the clustering takes into account the trajectory of the particles as a whole. 
This could be achieved by reading the trajectory file at each step and then assigning the clustered points to the closest cluster centroid. 
In FLEXDUST, future work should involve properly validating the ISRIC soil texture data that was implemented and tested out for the first time in this work. 
Moreover the implementation of different dust emission schemes in FLEXDUST would allow for investigating the sensitivity of the model setup to the choice of emission scheme. 
Lastly to make the FLEXPART and FLEXDUST model easier to use, a good online documentation in addition to the model development papers, on how to configure and setup the model should be created. 
In addition to a community portal where it is easy to ask questions and seek support within the FLEXPART community. 
These additions would be good step in right direction to bring the FLEXPART user experience on par with modern standards.

\section{Climatology of the East Asian dust cycle}

\emph{RQ: What is the main mode of deposition to the CLP? How are the source areas and dust transport paths among the sites different? How are the source areas different for the coarse and fine deposited dust?}

Whether the dust over the \acrshort{clp} is primarily deposited through wet- or dry deposition leads to different interpretations of the Chinese loess records.  
According to the FLEXPART/FLEXDUST model simulations, wet deposition is the main mode of deposition to the \acrshort{clp}. 
The differences between wet and dry deposition and the implications for the interpretation of the \acrshort{clp} will be discussed in \Cref{sec:wet_dry_characteristics}. 
In \Cref{sec:spatial_differences} the spatial differences in the source regions and dust transport trajectories between the receptor locations is discussed and how the spatial differences could be influenced by the topography of \acrshort{clp}. 
Lastly the spatial differences in the sources of the clay and silt sized dust are discussed in \Cref{sec:fine_vs_coarse}.

\subsection{Dry- versus wet deposition}\label{sec:wet_dry_characteristics}
The main mode of deposition to the \acrshort{clp}, according to the FLEXPART/FLEXDUST model simulations, is wet deposition and is the case for both the clay and silt particles. 
Compared to dry deposition, wet deposition is less influenced by the mean flow. This was shown both from the sensitivity experiments and the correlation analysis. 
The wet deposition at most of the locations was not significantly correlated with any circulation indices (except Lantian and Sacol).
Moreover, the correlation between wet deposition at the receptor and emissions was also weak. 
This implies that the deposition over the \acrshort{clp} is primarily episodic. 
A potential mechanism for how wet deposition could be important for the \acrshort{mar} to \acrshort{clp} must consider dust entrainment and precipitation. 
The passage of cold fronts is important for the formation of dust events. 
Precipitation typically forms ahead of or along the surface cold front as the warmer air is lifted over the denser cold air behind the cold front  \parencite{markowski2011mesoscale}. 
Moreover, as the cold front passes through the source regions, it generates strong winds at the surface, entraining the dust into the upper troposphere \parencite{li2015multi}. 
The dust can then be swiftly transported to the frontal rain band, efficiently scavenged by the clouds, and rained out. 

The dust transport trajectories from FLEXPART support this interpretation.
As seen from the increase in altitude of the trajectory, starting around 12 hours before approaching the receptor location consistent with the convective nature of the cold frontal rainband. 
Moreover, the more southerly orientation of the wet deposition trajectories suggests a larger contribution of moist southerly air masses. 
Conversely, the dry deposition trajectories have a larger northerly component, indicating more dry and cold air arriving at the receptor. 
Implying that if the northerly component is too strong, the cold front will move past \acrshort{clp}, and the frontal rainband will be situated to the south of the \acrshort{clp} hence no wet deposition.
Therefore, suggesting that wet deposition episodes are less likely to occur during the most intense dust storms. On the flip side, if the northerly winds are too weak to entrain a substantial amount of dust, then there will not be a large deposition event regardless if it rains or not.

The \acrfull{eam} predominately drives the seasonal shift in the circulation of this region. 
Therefore the wet and dry deposition trajectories can also be interpreted in terms of the transition from winter monsoon to summer monsoon, with the wind reversing direction from generally northerly in winter to southerly in summer. \textcite{dai2021define} divided the \acrshort{eam} into eight stages. 
The three stages of the \acrshort{eam} that are relevant for the dust transport is the late winter stage (February to early March), spring stage (March to late May) and pre-Meiyu (late May to late June). 
Predominately northwesterly winds characterise the late winter stage. 
Transitioning into spring, the southerly wind is strengthened, turning the northwesterly  winds into  predominately westerly winds. 
With the southerly shift in the predominant wind direction, the amount of precipitation is also enhanced. 
Finally, in the pre-Meiyu stage, the southerly wind is further enhanced, while the westerly and northerly winds are substantially weakened. 
Thus based on the predominately northwesterly transport indicated by the dry deposition trajectories, the conditions for strong dry deposition events are most favourable during late winter and early spring.
Conversely, strengthened southerlies in the spring and pre-Meiyu stages suggest that the conditions for wet deposition episodes are most favourable in spring and early summer.         

Still, until there is more observational evidence supporting wet deposition as the main mode of deposition to the \acrshort{clp}, it is more likely that the large contribution is due to the bias in the wet deposition, as discussed previously.
Even supposing that the ratio wet to dry deposition represents present-day deposition over the \acrshort{clp} it is necessary to keep in mind that the climate was very different during the glacial stages when the \acrshort{mar} over the \acrshort{clp} highest. 
Therefore it might be inaccurate to infer past deposition patterns from present-day climate.  

\subsection{Spatial variability of the dust sources to the CLP}\label{sec:spatial_differences}
Based on the FLEXPART/FLEXDUST simulations, the deserts North West of the \acrshort{clp} was the main source region of the dust deposited over the \acrshort{clp}. 
However, there are still noticeable differences in the source areas for the different sites. 
Generally, more dust was deposited at western sites compared to the eastern sites.  Shapotou, Yinchuan and SACOL, all located on the western side of the Liupan Mountains, have a large contribution from the Tengger and Badain Jaran desert. As opposed to Badoe, Luochuan and Lantian, which are more influenced by the deserts located on the Inner Mongolia Plateau. Suggesting that the Liupan Mountains could play a role in differentiating the source region to western and eastern \acrshort{clp}.
The topography could also explain some of the differences in source regions between SACOL and Lingtai. In contrast, SACOL experiences less dust transport from the northerly deserts and the opposite is the case for Lantian. This interpretation is also consistent with the centroid dust loading trajectories, which are more westerly than the other locations. 

The north south topography of the \acrshort{clp} offers a possible explanation of the difference in the deposition at Luochuan and Lantian. Luochuan and Lantian have very similar source regions and the amount of dry deposited material (\Cref{fig:source_contrib_2mmu}). 
However, Luochuan experiences much less wet deposition. The difference in elevation between Lantian and Luochuan is around 500 meters. 
As moisture is typically transported from the south, a significant portion of the precipitation might have precipitated before arriving at Luochuan. 
In fact, the precipitation at the ERA5 grid cell nearest to Luochuan is about 30\% less than Lantian. In addition, the wet deposition centroid dust loading trajectories for the two sites are nearly identical, suggesting that the same frontal systems are causing precipitation at both locations.  

\subsection{Fine versus coarse dust}\label{sec:fine_vs_coarse}
The spatial difference among the sites was largest for the silt-sized particles.
The main source regions of the coarse dust are located slightly further from the receptor location than fine dust. Moreover, while the deposited amount of fine dust does not vary that much between the sites. The deposition of coarse dust varies by more than one order of magnitude between Shapatou, the site experiencing the strongest deposition and Luochuan, the weakest deposition. This result is in agreement with the size distribution becoming finer along the transect from the north west to the south east \parencite{ding2000re,sun2003seasonal}.

As opposed to the silt particles, the clay particles could be transported within the lower troposphere even from the most remote sites (\Cref{fig:dust_loading_trajecs}). 
This could explain why dry deposition is more efficient for fine particles than coarse particles. Since the coarse dust entrained into the lower troposphere gets deposited over the source region before arriving at the receptor, the silt size particles have a stronger dependence on upper-level transport to reach the \acrshort{clp}. 
It makes it likely for the coarse dust to be scavenged by clouds and wet deposited, as evident from \Cref{fig:source_contrib_20mmu}h. The higher efficiency of wet deposition for the silt-sized particles could also be caused by the higher \acrshort{ccn} efficiency assumed in the simulations. However, the sensitivity wet deposition scheme in FLEXPART was not examined in this work.  

Possible processes that could carry the coarse dust into the upper troposphere, turbulent eddies, convection and mechanically forced lifting. 
The passage of a cold front destabilises the atmosphere producing favourable conditions for deep convection and strong turbulent eddies.    
Moreover, \textcite{yumimoto_elevated_2009} demonstrated that slopes of the \acrfull{ntp} could produce updraft winds strong enough to lift the dust from the Tarim Basin to a height between 9-12 km. Mechanical lifting might be the mechanism for responsible the localised area of high dust contribution over the Taklamakan in \Cref{fig:source_contrib_20mmu}.    

\section{Interannual variability of the East Asian dust cycle}
\emph{RQ: What are the differences in circulation patterns, dust transport paths and source regions between strong and weak deposition years? How does the interannual variability of winter circulation affect springtime dust?}
\subsection{Temporal variability of dust emission}

\subsection{Sources and transport in strong versus weak deposition years}
The composite analysis of the source contribution showed the sources  were more concentrated during strong than weak deposition years. 
Wet deposition was the main cause for the increase in the \acrshort{mar} to the \acrshort{clp} in strong deposition years. The more concentrated source contribution suggests that the strong deposition years are distinguished from the weak deposition years by a few strong predominately wet deposition events (\Cref{fig:source_contrib2mmu_anomalies} and \Cref{fig:source_contrib20mmu_anomalies}). 

Furthermore, looking at how the weak and strong deposition years differ in terms of their transport trajectories. 
The main difference in the  transport of dry deposited  dust between the strong and weak deposition years is the strengthening of northwesterly transport (see. \Cref{fig:strong_weak_drydepo_year_2mmu_trajecs} and \Cref{fig:strong_weak_drydepo_year_20mmu_trajecs}).
Moreover, the spread of the trajectories is slightly diminished with a small shift towards the north in strong deposition years. This is consistent with a stronger northerly component during strong deposition years. 
The wet deposition trajectory shows a strengthening of the transport during the strong deposition years like the dry deposition trajectories. 
However, the difference is less pronounced. 
Except for the transport of fine dust to Shapotau and coarse dust to SACOL, which has weaker transport during strong deposition years. 
Suggesting that intense dust storms might not be favourable conditions for wet deposition in these two cases.
Most of the locations had a more northerly trajectory during strong deposition events, the exception being SACOL. Suggesting that a strengthened northwesterly transport would produce unfavourable conditions for wet deposition at SACOL. 

   
\subsection{Linking winter circulation and spring deposition}
In the introduction of this thesis, I questioned the conclusion of \textcite{wyrwoll2016cold} and their claim to have demonstrated for the first time a close link between the occurrence of dust events and the strength of the \acrshort{eawm}. 
I had three core issues with their study: (1) The correlation between the \acrshort{eawm} and dust storm frequency was based on visibility measurements, which is more analogous to concentration.
(2) The frequency of \acrfull{caob} is used as a proxy for \acrshort{eawm}. However, \acrshort{caob}s occur due to the breakdown of the \acrfull{sh}, while a strong \acrshort{eawm} is characterised by a strengthened \acrshort{sh} \parencite{roe2009interpretation}. (3) They only considered the dust storm frequency; however, infrequent strong deposition events might contribute to the majority of the deposited dust. 

A comprehensive correlation analysis, including spring deposition and several indices representing the \acrshort{eawm} strength, was done to determine the influence of \acrshort{eawm} on spring deposition. 
The results showed no evidence for a significant influence of the strength of the \acrshort{eawm} on the deposition over the \acrshort{clp}. Contradicting the conclusions of \textcite{wyrwoll2016cold}.
While there was no link between the \acrshort{eawm} strength and deposition, the results showed a strong influence of the winter \acrshort{ao} on the strength of the spring dust deposition.    
As evident from the positive \acrshort{mslp} anomalies over the arctic shown 850hPa and \acrshort{mslp} composites anomalies in \Cref{fig:DJF_850_fine_composite} and \Cref{fig:DJF_850_coarse_composite}. 
While there was no apparent linkage between spring and the preceding winter in the low-level circulation composite anomalies. 
The 500hPa geopotential composite anomalies showed height anomalies in the winter preceding a strong deposition spring. 
Moreover, these anomalous winter conditions persisted into the following spring, with the anomaly being more prominent for the sites that had a strong correlation with the winter AO. 

\textcite{liu2018influence} examined the relationship between spring dust emissions and preceding winter \acrshort{ao} using the MERRA2 reanalysis.
They identified a similar pattern based on regression analysis between the winter 500hPa geopotential height and the leading \acrshort{eof} of springtime dust emissions. 
\textcite{liu2018influence} suggested that negative \acrshort{ao} produces anomalous cold conditions over central Siberia that can persist into the following spring. 
The effect of the cold conditions is amplified by snow-albedo and cloud feedbacks, producing favourable conditions for enhanced dust emissions in spring\parencite{liu2018influence}. 
The influence of the winter \acrshort{ao} on the spring surface temperature is confirmed by the correlation analysis, with a positive \acrshort{ao} corresponding to a reduced temperature gradient over East Asia (\Cref{fig:correlations}). 
Moreover, the deposition at the receptors all show a negative correlation between deposition and the spring temperature gradient, the same is true for the emissions.
However, the strength of the \acrshort{eawm}, while being significantly correlated with the winter temperature gradient, the \acrshort{eawm} have barely any influence on the spring temperature gradient. 
% The important role of the spring temperature on dust emissions is also in agreement with \textcite{liu2020impact}.

Furthermore, \textcite{yang2020interdecadal} examined the interdecadal variations in the paths of the \acrshort{caob}s influencing East Asia. 
They found a transition in the direction of the \acrshort{caob}s to favour a more northerly path since 1995.
Moreover, the maximum probability density distribution of the \acrshort{caob} paths since 1995 over Mongolia/Eastern Siberia (\textcite{yang2020interdecadal}, Figure 4)  coincide with the location of the negative winter 500hPa geopotential composite anomalies for SACOL and Lantian. In addition, \textcite{yang2020interdecadal} showed that the northerly \acrshort{caob} path was associated with negative \acrshort{ao}-like composite anomalies and thus consistent SACOL and Lantian being the two sites strongest influenced by the \acrshort{ao}. Compared to wet deposition at Shapotou is only weakly influenced by the  winter \acrshort{ao}, comparing the location of the winter 500hPa geopotential height anomaly with the \acrshort{caob} path distribution, Shapotou seems to be more influenced by the westerly cold surge path, common before 1995.       
This also illustrates another issue with the evidence for the conclusions of \textcite{wyrwoll2016cold}, since in addition to the frequency and intensity of the \acrshort{caob}s, the path of the \acrshort{caob} can also influence the \acrshort{mar} to the \acrshort{clp}.  
% A weaker yet stream does also make it more favourable for cold air intruding into East Asia \parencite{he2017impact}. 
% Negative \acrshort{ao} is also associated with increased snow and cloud cover over central Siberia, increasing the albedo amplifying the surface cooling. 

% Moreover, the cold conditions strengthen the temperature gradient between Mongolia and northern China, causing stronger northwesterly winds. 
% This is consistent with the spring composite anomalies of EOF1 in \Cref{fig:eof_composite_MAM}a which shows anomalous cyclonic circulation over Central Siberian and strengthened North westerlies. 
% However from the spring time composite anomalies for the total deposition  \Cref{fig:MAM_850_fine_composite} and \Cref{fig:MAM_850_coarse_composite} only Yinchuan similar circulation anomalies to the emissions. 

% Comparing the spring EOF2 circulation composite anomalies with the deposition circulation composite anomalies, SACOL fine clay \Cref{fig:MAM_850_fine_composite}e show similar circulation anomalies to that of positive EOF2 years. Accordingly strong deposition of fine dust at SACOL suggests a large contribution from Taklamakan, this is also consistent with source contribution composite anomalies in \Cref{fig:source_contrib2mmu_anomalies}. Interestingly Luochuan coarse silt show circulation anomalies more in line with the negative EOF2 years. 

% \subsection{Summary}

\section{Implication for interpreting loess records}

Comparisons of Zircon U-Pb geochronologies from the \acrshort{clp} and the surrounding source regions have revealed the Taklamakan, Western Mu Us and Northern Tibetan Plateau as the most likely source for loess at the \acrshort{clp} \parencite{bird2015quaternary}. Due to its distance from the deposition centre and the dominant medium to coarse silt composition of the loess, Taklamakan is often disregarded as a major source to the \acrshort{clp}. However    
dwe

Therefore not related to the strength of the winter monsoon winds often assumed in paleoclimate studies \parencite{chen2006zr}. 
Thus if model simulations are representative of the actual ratio of wet to dry deposition over the \acrshort{clp}, it would mean a fundamental change in how the Loess records are interpreted.