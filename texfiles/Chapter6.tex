
\Chapter{}{Discussion}\label{chap:Discussion}
I presented the results from the FLEXDUST and FLEXPART simulation in the previous chapter, where I showed that the two models could reasonably reproduce the emission, transport and dry deposition of East Asian dust.
Despite a good agreement between the model and observed dry deposition flux, there was less compliance between the model and the observed wet deposition flux. 
The impact of model uncertainties for the interpretation and analysis of the results will be discussed in \Cref{sec:model_uncertainties}. 
% The consequences of the low agreement between model and observation on the  wet deposition flux  and other model uncertainties for the interpretation and analysis of the model results will be discussed in \Cref{sec:model_uncertainties}. 
The results presented in the last chapter also showed spatial differences in the multiyear mean transport trajectories and source contribution for the different receptor locations. 
These differences will be discussed further in \Cref{sec:spatial_differences}. 
Moreover, the results also showed a disproportionally large contribution of wet deposited  dust to the total deposition at all the sites and particle size. 
The implication of this result will be discussed in \Cref{sec:wetdep_discussion}. 
The results suggested a strong influence from the preceding winter \acrshort{ao} on dust emissions and deposition. 
However, the more striking result was the weak influence of the \acrshort{eawm} strength on the deposition in disagreement with the conclusion of \textcite{wyrwoll2016cold}.
Therefore, the linkage between winter circulation and springtime deposition will be discussed in the \Cref{sec:linkage_winter_dust}. 
I will end by highlighting the implications of these results for the interpretation of the loess record.  

% The previous chapter the results from the FLEXPART and FLEXDUST simulations. When compared to observations FLEXPART/FLEXDUST performed well, especially for the dry deposition. However, the wet deposition did not compare that well to the observations. Conversely, the uncertainties related to the wet deposition, and how this would impact the results will be discussed first in \Cref{sec:model_uncertainties}.
% and  model setup was shown to be able to represent the 
%   Start with a few sentences that summarize the most important results.
% What are the major patterns in the observations? (Refer to spatial and temporal variations.)
% What are the relationships, trends and generalizations among the results?
% What are the exceptions to these patterns or generalizations?
% What are the likely causes (mechanisms) underlying these patterns resulting predictions?
% Is there agreement or disagreement with previous work?
% Interpret results in terms of background laid out in the introduction - what is the relationship of the present results to the original question?
% What is the implication of the present results for other unanswered questions in earth sciences, ecology, environmental policy, etc....?
% Multiple hypotheses: There are usually several possible explanations for results. Be careful to consider all of these rather than simply pushing your favorite one. If you can eliminate all but one, that is great, but often that is not possible with the data in hand. In that case you should give even treatment to the remaining possibilities, and try to indicate ways in which future work may lead to their discrimination.
% Avoid bandwagons: A special case of the above. Avoid jumping a currently fashionable point of view unless your results really do strongly support them. 
% What are the things we now know or understand that we didn't know or understand before the present work?
% Include the evidence or line of reasoning supporting each interpretation.
% What is the significance of the present results: why should we care? 
\section{Discussrion from results}

This suggests that the transport of coarse dust from the remote source regions have to involve upper tropospheric transport. 
This is also consistent with the coarse dust being transported more swiftly.  
Moreover, this also approximates how long before deposition the dust would have to be entrained into the atmosphere, assuming that the dust must be entrained before entering the frontal region.  

as expected based on the hypothesis that the coarse dust is particles mainly transported by faster-moving air in the upper troposphere.

The weaker \acrshort{eawm} produces weaker northwesterly winds, favouring cold air intrusions into the Tarim Basin.  

The easterly wind anomalies could explain the reduction in emissions over Mongolia. 

The more efficient dust transport in strong deposition years is consistent with the dust being transported at higher altitudes as shown in \Cref{fig:strong_weak_drydepo_year_20mmu_trajecs}h and more intense winds.

During strong deposition years, the more concentrated source contribution suggests that the strong deposition years are set apart from the weak deposition years by having a large contribution from rare strong dust events. 

\section{TODO:}


\begin{itemize}
    \item The solution that has evolved over time is to set up the Discussion section as a "dialogue" between Results and Theories -- yours and everyone elses'. 
    \item Begin with a restatement of your research question, followed by a statement about whether or not, and how much, your findings "answer" the question.  These should be the first two pieces of information the reader encounters.
    \item Define research question
    \item I haven't been there
\end{itemize}



\section{}

\section{The spatial difference in the sources of dust deposition over CLP}\label{sec:spatial_differences}
Based on the FLEXPART/FLEXDUST simulations the deserts North West of the \acrshort{clp} was found to be the main source region of the dust deposited over the \acrshort{clp}. However, there is still noticeable differences in the source areas for the different sites. Generally more dust were deposited at western sites compared to the eastern sites.  Shapotou, Yinchuan and SACOL, which are all located on the western side of the Liupan Mountains, has a large contribution from the Tengger and Badain Jaran desert as opposed to Badoe, Luochuan and Lantian, which are more influenced by the deserts located on the Inner Mongolia Plateau. The topography could also explain some of the difference in source regions between SACOL and Lingtai, whereas SACOL experience less dust transport from the northerly deserts and the opposite is the case for Lantian. This interpretation is also consistent with the centroid dust loading trajectories, which are in a more easterly direction compared to the other locations. 

The north south topography of the \acrshort{clp} might explain in the difference in the amount of deposition at Luochuan and Lantian. Luochuan and Lantian have very similar source regions and the amount of dry deposited material (\Cref{fig:source_contrib_2mmu}). However, Luochuan experience much less wet deposition. The difference in elevation between Lantian and Luochuan is around 500 meters. As moisture is typically transported from the south, a significant portion of the precipitation might have precipitated before arriving at Luochuan. In fact, the precipitation at the ERA5 grid cell nearest to Luochuan is about 30\% less than Lantian. In addition, the wet deposition centroid dust loading trajectories for the two sites are nearly identical, suggesting that the same frontal systems are causing precipitation at both locations.              


%  Which is consistent previous modelling studies \textcite{shi2011distinguishing}.

\section{Transport of fine and coarse dust to the CLP}
The model result suggested a stronger contribution from remote sources for the site located at the western part of the \acrshort{clp} compared to the eastern sites. This spatial difference was largest for the silt particles. The   
Opposed to the silt particles, the clay particles could be transported within the lower troposphere even from the most remote sites (\Cref{fig:dust_loading_trajecs}).  For the silt size dust to be able to reach the \acrshort{clp} from the remote source regions, the dust has to be entrained much higher into the atmosphere. There are three possible mechanisms that could carry the dust into the upper troposphere, (1) turbulent eddies, (2) convection and (3) mechanically forced lifting. The two first mechanisms predominantly dependent on the atmospheric stability. In the Tarim Basin \textcite{yumimoto_elevated_2009} demonstrated slopes of the \acrfull{ntp} could produce updraft winds strong enough to lift the dust veil to height between 9-12 km. 

Comparisons of Zircon U-Pb geochronologies from the \acrshort{clp} and the surrounding source regions have revealed Taklamakan, Western Mu Us and Northern Tibetan Plateau as the as the most likely source for loess at the \acrshort{clp} \parencite{bird2015quaternary}. Due to it's distance from the depositioncentre and the dominant medium to coarse silt composition of the loess, Taklamakan is often disregarded as a major source to the \acrshort{clp}.                 




\section{Mechanism for the interannual variation of dust deposition over CLP }
The polar cold air outbreaks are well known to be the driving factor of dust production in North East Asia. The intensity of the winter monsoon is often thought to regulate the cold air outbreak activity, the analysis of interannual variation in dust deposition over \acrshort{clp} pointed out a very weak correlation between \acrshort{eawm} strength and  


\subsection{Link between AO and dust deposition over CLP?}
The winter circulation composite anomalies and the correlation analysis showed a strong connection between the winter AO and dust deposition. 
% During negative AO years the jetstream is weaker causing it to meander further south. 
A weaker yet stream does also make it more favourable for cold air intruding into East Asia \parencite{he2017impact}. 
Negative \acrshort{ao} is also associated with a increased snow and cloud cover over central Siberia which increases the albedo amplifying the surface cooling. 
\textcite{liu2018influence} examined the relationship between spring dust emissions and preceding winter \acrshort{ao} using the MERRA2 reanalysis showed that the anomalously cold surface conditions over Central Siberia that arise during negative \acrshort{ao} years can produce anomalously cold conditions that last into spring. 
Moreover the cold conditions strengthens the temperature gradient between Mongolia and northern China causing stronger than usual north westerlies. 
This is consistent with the spring composite anomalies of EOF1 in \Cref{fig:eof_composite_MAM}a which shows anomalous cyclonic circulation over Central Siberian and strengthened North westerlies. 
However from the spring time composite anomalies for the total deposition  \Cref{fig:MAM_850_fine_composite} and \Cref{fig:MAM_850_coarse_composite} only Yinchuan similar circulation anomalies to the emissions. 

Comparing the spring EOF2 circulation composite anomalies with the deposition circulation composite anomalies, SACOL fine clay \Cref{fig:MAM_850_fine_composite}e show similar circulation anomalies to that of positive EOF2 years. Accordingly strong deposition of fine dust at SACOL suggests a large contribution from Taklamakan, this is also consistent with source contribution composite anomalies in \Cref{fig:source_contrib2mmu_anomalies}. Interestingly Luochuan coarse silt show circulation anomalies more in line with the negative EOF2 years.    

\section{Implication for interpreting loess records}

\section{Model uncertainties and wet deposition bias}\label{sec:model_uncertainties}
Uncertainties and model limitations are embedded in any modelling study, and considering these uncertainties is an important piece of interpreting the results.  
The sensitivity experiments and model evaluation highlighted some of the limitations of FLEXPART/FLEXDUST. 
In particular, the inability of FLEXPART to reproduce the variability in the wet deposition flux compared to observations. 
Achieving a good performance of the wet deposition scheme is particularly important, given that wet deposition was the primary mode of deposition according to the FLEXPART simulations. 
The inclusion of only four size bins could be a factor that could affect the performance of the wet deposition scheme.
However, the sensitivity of the wet deposition scheme to the particle size was not considered in this thesis.
\textcite{flexdust_ref_2016} achieved relatively good agreement between FLEXDUST combined with a forward FLEXPART simulations and observations of annual dust deposition fluxes using 10 size bins. 
% Moreover, they kept the particles 120 days rather than just 5 days as in this case, but based on the analysis of \textcite{osada2014wet} 5 days should be sufficient for the dust to be carried from the East Asian  sources  to the Japanese coast. 
Another possible reason for why \textcite{flexdust_ref_2016} achieved good agreement with observations is that they compared annual deposition fluxes, and it is possible and that the apparent biases on the monthly level compensate each other such that the total deposition is around the observed value. 
Besides \textcite{flexdust_ref_2016}, there have not been other studies that have evaluated the performance of FLEXPART to simulate deposition of mineral dust.
However, there have been more studies done comparing simulated deposition of black carbon to with observations, e.g. \textcite{flexpart_wetdep, choi_investigation_2020}. 
\textcite{choi_investigation_2020} suggested that FLEXPART was not able to capture the regional variability due to severely underestimating the in-cloud scavenging coefficient. Moreover, by diagnosing the relative importance of the input variables 
they found that the convective available potential energy (CAPE), should be considered as an important parameter to improve the in-cloud scavenging scheme. 
Moreover, there is also a known issue with the current wet deposition scheme related to the interpolation and disaggregation of the input variables \parencite{tipka2021effects}. 
I did not have the time to test with the new interpolation scheme in FLEXPART. 
That would require preparing all the forcing data again, and support for the new interpolation scheme has not yet officially been released.

Another factor that disproportional impacts the estimated wet deposition, is the spatial and temporal distribution of the dust emissions simulated in FLEXDUST. 
As shown from the input forcing sensitivity experiment, the accumulated source contribution showed a higher sensitivity to the input forcing than the FLEXDUST accumulated emission flux and FLEXPART emission sensitivity. Given how episodic the wet depositions events are and the dependence of not only emissions in the source region, but also precipitation at receptor location following the dust event. Slight changes in the temporal and spatial evolution of the dust emissions can have a large effect on the amount of dust deposited at the receptor. Moreover the analysis showed that the strong deposition events did not necessarily coincide with the strongest dust events. The agreement between ERA5 and ERA-Interim FLEXDUST simulations were higher for the strong dust events compared to the weaker dust events. Because the large synoptic scale dust events are better represented in the forcing data. The resulting bias towards strong dust events is less important for the dry deposition flux since the dry deposition strongly correlated with the dust emissions as shown in \Cref{fig:correlations}. 

% Lastly, it is also important to keep in mind that there are also uncertainties in the observations \parencite{osada2014wet}. 

Then this leads back to the initial question, how does the apparent biases in the wet deposition affect the analysis? 
The important factor that FLEXPART has to get right is the interannaul variability in the amount of springtime deposited dust to the \acrshort{clp}. 
Since this is the premise for the composite analysis. 
Based on the two years of data used for validation here, FLEXPART does agree with the observation on the year to year difference. 
However, from only two years of data it is hard to say for certain that this generally the case. 
Therefore, the wet deposition is a caveat of this study and is something that should be improved upon in the future. 

\section{Model assessment}
In the more than two years that I have been working on this thesis, a considerable amount on time have been spent on understanding the model source code, physical parameterisation, finding the correct model set up and developing scripts to analyse the model results. 
Based on what I have learned and experienced while working with FLEXPART and FLEXDUST, I will in this section discuss the  
advantages and disadvantages of this model set up for dust modelling applications, and give some suggestions where the model should be improved.
\subsection{FLEXPART}
\subsection{FLEXDUST}
\subsection{Wet deposition bias}\label{sec:model_uncertainties}
Uncertainties and model limitations are embedded in any modelling study, and considering these uncertainties is an important piece of interpreting the results.  
The sensitivity experiments and model evaluation highlighted some of the limitations of FLEXPART/FLEXDUST. 
In particular, the inability of FLEXPART to reproduce the variability in the wet deposition flux compared to observations. 
Achieving a good performance of the wet deposition scheme is particularly important, given that wet deposition was the primary mode of deposition according to the FLEXPART simulations. 
The inclusion of only four size bins could be a factor that could affect the performance of the wet deposition scheme.
However, the sensitivity of the wet deposition scheme to the particle size was not considered in this thesis.
\textcite{flexdust_ref_2016} achieved relatively good agreement between FLEXDUST combined with a forward FLEXPART simulations and observations of annual dust deposition fluxes using 10 size bins. 
% Moreover, they kept the particles 120 days rather than just 5 days as in this case, but based on the analysis of \textcite{osada2014wet} 5 days should be sufficient for the dust to be carried from the East Asian  sources  to the Japanese coast. 
Another possible reason for why \textcite{flexdust_ref_2016} achieved good agreement with observations is that they compared annual deposition fluxes, and it is possible and that the apparent biases on the monthly level compensate each other such that the total deposition is around the observed value. 
Besides \textcite{flexdust_ref_2016}, there have not been other studies that have evaluated the performance of FLEXPART to simulate deposition of mineral dust.
However, there have been more studies done comparing simulated deposition of black carbon to with observations, e.g. \textcite{flexpart_wetdep, choi_investigation_2020}. 
\textcite{choi_investigation_2020} suggested that FLEXPART was not able to capture the regional variability due to severely underestimating the in-cloud scavenging coefficient. Moreover, by diagnosing the relative importance of the input variables 
they found that the convective available potential energy (CAPE), should be considered as an important parameter to improve the in-cloud scavenging scheme. 
Moreover, there is also a known issue with the current wet deposition scheme related to the interpolation and disaggregation of the input variables \parencite{tipka2021effects}. 
I did not have the time to test with the new interpolation scheme in FLEXPART. 
That would require preparing all the forcing data again, and support for the new interpolation scheme has not yet officially been released.

Another factor that disproportional impacts the estimated wet deposition, is the spatial and temporal distribution of the dust emissions simulated in FLEXDUST. 
As shown from the input forcing sensitivity experiment, the accumulated source contribution showed a higher sensitivity to the input forcing than the FLEXDUST accumulated emission flux and FLEXPART emission sensitivity. Given how episodic the wet depositions events are and the dependence of not only emissions in the source region, but also precipitation at receptor location following the dust event. Slight changes in the temporal and spatial evolution of the dust emissions can have a large effect on the amount of dust deposited at the receptor. Moreover the analysis showed that the strong deposition events did not necessarily coincide with the strongest dust events. The agreement between ERA5 and ERA-Interim FLEXDUST simulations were higher for the strong dust events compared to the weaker dust events. Because the large synoptic scale dust events are better represented in the forcing data. The resulting bias towards strong dust events is less important for the dry deposition flux since the dry deposition strongly correlated with the dust emissions as shown in \Cref{fig:correlations}. 

% Lastly, it is also important to keep in mind that there are also uncertainties in the observations \parencite{osada2014wet}. 

Then this leads back to the initial question, how does the apparent biases in the wet deposition affect the analysis? 
The important factor that FLEXPART has to get right is the interannaul variability in the amount of springtime deposited dust to the \acrshort{clp}. 
Since this is the premise for the composite analysis. 
Based on the two years of data used for validation here, FLEXPART does agree with the observation on the year to year difference. 
However, from only two years of data it is hard to say for certain that this generally the case. 
Therefore, the wet deposition is a caveat of this study and is something that should be improved upon in the future. 

The ability of FLEXPART/FLEXDUST to directly establish which dust sources that are contributing to the dust deposition, makes this model set up extremely powerful for better understanding the loess records. If determining the source contribution is what you are after, then FLEXPART/FLEXDUST is a very competent model. The model can also be relatively computationally efficient, as long as there is not too many receptor points.
Unfortunately it is not the most convenient or easy to use model out there.   

As described in the in \Cref{chap:method_trajec_analysis}, the particle trajectory clusters available in the FLEXPART output was not suitable for diagnosing the dust transport trajectories.
The lack of a suitable trajectory clustering routine puts FLEXPART at a bit of an disadvantage , compared to for instance the similar HYSPLIT model \parencite{draxler2010hysplit}.  

Improving the clustering routine in FLEXPART would involve doing clustering based on the trajectory as a whole. This could be achieved by reading in the trajectory file at each time step and then assign the clustered points to the closest cluster centriod. The most accurate approach would be to store the trajectory of each particle either in a file or in memory and then re-cluster the trajectories at each time step. However the performance penalty of this approach might be to large.

I would argue that the current format of the backward emissions sensitivity is not optimal. As described in \Cref{sec:flexpart} the deposition and source contribution is obtained by multiplying the emissions sensitivity with the emission strength. However the current output format has the dimmensionallity of (pointspec, time, height, lon, lat) where the time dimension starts and the end of the simulation a  then goes backward to the beginning. The pointspec dimension represent the forward time dimension which corresponding to a new particle release. The length of the backward trajectories are then controlled by specifying the maximum age through the AGECLASS parameter. First of all the current format result in a lot of empty entries as when the maximum age is exceeded. In therms of the ease of use of the output from backward FLEXPART simulations the output format is a major problem and can easily cause miss understandings. The current output format is designed to be same regardless of the whether it done using a forwad or backward approach. However I belive that FLEXPART would be easier to use if an output format specifically for the backward simulation were to be included. Which had the dimensions of (btime, time, height, lon,lat) where btime correpsond to the time along the back trajectory in seconds and time would be the forward time dimensions. Then the output from FLEXPART could be use directly without having to post processes first. 
thedustfluxwre

\