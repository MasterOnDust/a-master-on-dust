
\subsection{FLEXPART}
FLEXPART (Flexible Particle dispersion model) is a lagrangian particle dispersion model (LPDM). 

\subsubsection{What is Lagrangian particle dispersion model?}
LPDM can either be run forward or backwards in time. Forward simulation, are a
natural choice for studying the dispersion of tracers from known sources.
Forward simulation is natural choice when examining dispersion of an atmospheric
tracer from a source point of view, for example studying the dispersion of the
ash plume from a vulcanic eruption or the transport of radio active material
after a nuclear accident. When particles are followed backwards in time they create a so 
called retro plume. 
\subsubsection{FLEXPART wet depostion}
Wet deposition in FLEXPART is calculated by releasing particles through out the
whole atmospheric column. To differentiate between in-cloud and below-cloud
removal in the model, require to know the position of the computational FLEXPART
particle relative to the cloud. Since FLEXPART V10 there has been the option to
use 3D cloud total water content, available in the ECMWF forcing
\parencite{flexpart_wetdep}.  

Above the cloud no scavenging can occur, inside the cloud aerosols can be
scavenged by acting as nucleation sites for cloud droplets or
ice crystals (nucleation scavenging), below the cloud, aerosols can be washed-out
through interactions with precipitation as the precipitation falls to the ground
(impaction scavenging). The efficiency of the wet scavenging for insoluble aerosols 
depends primarily on the size of the particle. Particularly for below cloud 
scavenging, which is not very efficient for submicron particles as they tend to
follow the streamlines. Initially mineral dust is insoluble and not very efficient CCN,
however by weathering during transport the mineral dust aerosols might acquire a coating of
of some soluble material e.g. sulfate \textcite{Dust_aerosols_coating2001}. Both in-cloud
(\verb|PCCN_AERO| and \verb|PIN_AERO|) and below cloud (\verb|PCRAIN_AERO| and \verb|PCSNOW_AERO|),  
scavenging parameters, can be adjusted in the \verb|SPECIES_nnn| file to fit the
modeled species.     

Precipitation does not necessarily occur over whole grid cell,
FLEXPART uses an empirical relation to get the sub-grid precipitation intensity
$ I $ by scaling the  the surface precipitation intensity ($ I_t $) in the ECMWF
input data, described in detail in \textcite{Flexpart-2005_ref_paper}.   

% The wet deposition
% in FLEXPART is computed by releasing computational FLEXPART particles in the
% entire atmospheric column.  