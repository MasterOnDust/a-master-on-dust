
\subsection{FLEXPART}

\subsubsection{FLEXPART wet depostion}

The wet deposition scheme in FLEXPART include both in cloud and below cloud
removal. To determine wether a computational particle residing in a column with
precipitation is situated above, inside or below a cloud FLEXPART uses
meteorological information from the ECMWF input data. Since FLEXPART V10 the
vertical extent of the cloud can be determined by the specific cloud liquid
water content (CLWC) and specific cloud ice water content or from the summed
quantity cloud total water content (CTWC) \parencite{flexpart_wetdep}. If the
particle is found to be either inside or below a cloud the scavenging
coefficient in FLEXPART is determined by either by the in-cloud scheme or the
below cloud scheme.  

Since the precipitation does not necessarily occur over whole grid cell,
FLEXPART uses an empirical relation to get the sub-grid precipitation intensity
$ I $ by scaling the  the surface precipitation intensity ($ I_t $) in the ECMWF
input data, described in detail in \textcite{Flexpart-2005_ref_paper}.   

Above the cloud no scavenging can occur, in side the cloud aerosols can be
scavenged by acting as nucleation sites for cloud droplets or
ice crystals (nucleation scavenging), below the cloud aerosols can be washed-out
through interactions with precipitation as the precipitation falls to the ground
(impaction scavenging). 

In FLEXPART nucleation scavenging is only activated for particles residing at
altitudes where cloud water is present and in the fraction of the gridcell where
precipitation is occurring. The    

% The wet deposition
% in FLEXPART is computed by releasing computational FLEXPART particles in the
% entire atmospheric column.  